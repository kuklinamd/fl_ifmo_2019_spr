\documentclass[12pt, a4paper] {ncc}
\usepackage[utf8] {inputenc}
\usepackage[T2A]{fontenc}
\usepackage[english, russian] {babel}
\usepackage[left=2cm,right=2cm,top=2cm,bottom=2cm,bindingoffset=0cm]{geometry}
\usepackage{xcolor}
%\usepackage{bussproofs}
\usepackage{syntax}
\usepackage{listings}
\usepackage{proof}

\begin{document}
\setcounter{figure}{0}

\section{Конкретный синтаксис.}
\begin{grammar}
<ident> ::= <nondigit> \alt <nondigit> <alphanum> \alt <nondigit> <alphanum> <ident>

<binop> ::= `+' | `.-' | `^' | `*' | `/' | `==' | `/= ' | `<=' | `<' | `>' | `>=' | `|' | `\&\&'

<unop> ::= '-' | '!'

<num-lit> ::= number

<bool-lit> ::= 'T' | 'F'

<lit> ::= <num-lit> | <bool-lit>

<constr-name> ::= <capital-alpha> | <capital-alpha> <ident>

<pat> ::= <var>
\alt <lit>
\alt <constr-name>
\alt <constr-name> <pats>
\alt '(' <pat> ')'

<pats> ::= <pat> | <pat> <pats>

<expr> ::= <ident>
\alt <lit>
\alt <app>
\alt 'if' <expr> 'then' <expr> 'else' <expr>
\alt 'let' var '=' <expr> 'in' <expr>
\alt 'case' '(' <expr> ')' 'of' '\{' <case-body> '\}'
\alt  <expr> <binop> <expr>
\alt <unop> <expr>
\alt '(' <expr> ')'

<app> ::=  <ident> ' ' <ident>
\alt <ident> ' ' <lit>
\alt <ident> ' ' '(' <expr>')'
\alt '(' <expr> ')' <ident>
\alt '(' <expr> ')' <lit>
\alt '(' <expr> ')' '(' <expr> ')'

<case-entry> ::= '|' <pat> '->' <expr> 

<case-body> ::= <case-entry> | <case-entry> <case-body>

<bind> ::= <ident> (<arg> | $\varepsilon$) '=' <expr>

<arg> ::= <ident> | <ident> ' ' <arg>

<data> ::= 'data' <constr-name> '=' <data-body>

<data-body> ::= <data-entry> | <data-entry> '|' <data-body>

<data-entry> ::= <constr-name> | <constr-name> <pats>

<type-def> ::= ':' ident ':' <type>

<type> ::= 'Int'
\alt 'Bool'
\alt <type> '->' <type>

<decl> ::= <bind> | <data> | <type-def>

<decls> ::= <decl> ';' | <decl> ';' <decls>

\end{grammar}

Введены две новые синтаксические категории: объявление типа функций (\texttt{type-def})
и объявление типов данных (\texttt{data}).

Каждой функции должно предшествовать объявление её типа.

Примеры:
\begin{enumerate}
\item Объявление типы данных.
\begin{lstlisting}[language=Haskell]
data EitherBI = Left Bool | Right Int;
\end{lstlisting}

\item Объявление типа функции.
\begin{lstlisting}[language=Haskell]
:f: Int -> Int;
\end{lstlisting}
\end{enumerate}

Добавлена синтаксическая конструкция \texttt{case}.
\begin{lstlisting}[language=Haskell]
:f: EitherBI -> Bool
f x = case (x) of {
    | Right b -> b
    | Left i  -> T
};
\end{lstlisting}

И паттерн-матчинг.
\begin{lstlisting}[language=Haskell]
:f: EitherBI -> Bool
f (Right b) = b;
f (Left i) = T;
\end{lstlisting}

Правила проверки типов тогда сводятся к проверке непротиворечивости
объявления шаблона и его использовании.

Правила типизации шаблонов.

\infer{\text{PatLit x} : \text{Int}}{
	x \in \text{Int}
}

\ \\
\infer{\text{PatBool x} : \text{Bool}}{
	x \in \text{Bool}
}

\ \\
\infer{\Delta \vdash \text{PatVar x} : A}{}

\ \\
\infer{\Delta \text{PatConstr D } \{a_i\}_{i \in 1 \ldots n} : A}{
	\Delta D : \{A_i\}_{i \in 1 \ldots n} \to A &
	\Delta \vdash \{a_i\}_{i \in 1 \ldots n} : \{A_i\}_{i \in 1 \ldots n}
}

Тогда case будет типизироваться как \\

\infer{\Gamma, \Delta \vdash \text{case } x \text{ of } \{p_i \to e_i\}_{i \in 1 \ldots n}  : B} {
	\Gamma, \Delta \vdash x : A &
	\Delta \vdash \{p_i\}_{i \in 1 \ldots n} : A &
	\Gamma, \Delta \vdash \{e_i\}_{i \in 1 \ldots n} : B
}

\section{Абстрактный синтаксис}

Представлен в виде АСД.

\begin{lstlisting}[language=Haskell]
type Ast = [Decl]

data Decl = BindDecl Bind | DataDecl Data | TypeDecl Name Type

data Data = Data Name [Constr]

data Constr = Constr Name [Name]

data Type = Arrow Type Type | TInt | TBool | TData Name

data Bind = Bind Name [Pat] Expr

data Pat = PatVar Name | PatCtr Name [Pat] | PatLit Lit

data Expr = Var Name
          | Lit Lit
          | App Expr Expr
          | If Expr Expr Expr
          | Case Expr [(Pat, Expr)]
          | Let Name Expr Expr
          | UnOp UnOperator Expr
          | BinOp BinOperator Expr Expr

data Lit = ILit Integer | BLit Bool

\end{lstlisting}

\end{document}

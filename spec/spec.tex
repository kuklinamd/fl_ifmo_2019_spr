\documentclass[12pt, a4paper] {ncc}
\usepackage[utf8] {inputenc}
\usepackage[T2A]{fontenc}
\usepackage[english, russian] {babel}
\usepackage[left=2cm,right=2cm,top=2cm,bottom=2cm,bindingoffset=0cm]{geometry}
\usepackage{xcolor}
%\usepackage{bussproofs}
\usepackage{syntax}
\usepackage{listings}

\begin{document}
\setcounter{figure}{0}

\section{Конкретный синтаксис.}
\begin{grammar}
<ident> ::= <nondigit>
\alt <nondigit> <ident_rest>

<ident_rest> ::= <alphanum> | <alphanum> <ident_rest>

<cmp-op> ::= `==' | `/= ' | `<=' | `<' | `>' | `>=' 
<md-op> ::= `*' | `/' 
<pm-op> ::= `+' | `.-' 

<unop> ::= '-' | '!'

<num-lit> ::= number

<bool-lit> ::= 'T' | 'F'

<lit> ::= <num-lit> | <bool-lit>

<expr> ::= <binop-expr>

<binop-expr> ::= <or-expr>

<or-expr> ::= <and-expr> `||' <or-expr> | <and-expr>

<and-expr> ::= <cmp-expr> `&&' <and-expr> | <cmp-expr>

<cmp-expr> ::= <pm-expr> <cmp-op> <pm-expr> | <pm-expr>

<pm-expr> ::= <pm-expr> <pm-op> <md-expr> | <md-expr>

<md-expr> ::= <md-expr> <md-op> <pow-expr> | <pow-expr>

<pow-expr> ::= <unop-expr> `^' <pow-expr> | <unop-expr>

<unop-expr> ::= <un-op> <lit>
\alt <un-op> <var>
\alt <un-op> `(' <atom-expr> `)'

<atom-expr> ::= <ident>
\alt <lit>
\alt <app>
\alt `if' <expr> `then' <expr> `else' <expr>
\alt `let' var `=' <expr> `in' <expr>
\alt  <expr> <binop> <expr>
\alt <unop> <expr>
\alt `(' <expr> `)'

<app> ::= <ident> ` ' <app-args>
\alt `(' <expr> `)' ` ' <app-args>

<app-args> ::= <app-arg> | <app-arg> ` ' <app-args>

<app-arg> ::= <lit> | <ident> | `(' <expr> `)'

<bind> ::= <ident> <arg> '=' <expr>
\alt <ident> '=' <expr>

<arg> ::= <ident> | <ident> ' ' <arg>

<decl> ::= <bind> ';' <decl> | <bind> ';'

\end{grammar}

Примеры.
\begin{enumerate}
\item Объявление функции/переменной.
\begin{lstlisting}[language=Haskell]
f x = x;
n = 10;
\end{lstlisting}
\item Использование условного оператора \texttt{if}
\begin{lstlisting}[language=Haskell]
f x = if x == 0 then 10 else 20 + x;
\end{lstlisting}

\item Использование \texttt{let}-связывания.
\begin{lstlisting}[language=Haskell]
f y = let x = 10 * y in x ^ x;
\end{lstlisting}
\item Вызов функции
\begin{lstlisting}[language=Haskell]
f x = x;
g = f 10;

ff x y = x + y;
gg = ff 10 20;
\end{lstlisting}
\end{enumerate}

\section{Абстрактный синтаксис}

Представлен в виде АСД.

\begin{lstlisting}[language=Haskell]
type Ast = [Decl]

data Decl = BindDecl Bind

data Bind = Bind Name [Name] Expr

data Expr = Var Name
          | Lit Lit
          | App Expr Expr
          | If Expr Expr Expr
          | Let Name Expr Expr
          | UnOp UnOperator Expr
          | BinOp BinOperator Expr Expr

data Lit = ILit Integer | BLit Bool

\end{lstlisting}

\end{document}
